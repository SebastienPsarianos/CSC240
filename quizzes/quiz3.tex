\documentclass[11pt]{article}
\usepackage{amssymb}
\usepackage{fullpage}
\pagestyle{empty}
\setlength{\parskip}{1ex}
\setlength{\parindent}{0in}

\def\nats{{\mathbb N}}
\def\ints{{\mathbb Z}}
\newcommand{\Implies}{\mbox{ IMPLIES }}
\newcommand{\And}{\mbox{ AND }}
\newcommand{\Iff}{\mbox{ IFF }}
\newcommand{\Or}{\mbox{ OR }}
\newcommand{\Not}{\mbox{NOT }}

\begin{document}

{\bf \large Quiz 3}

\noindent
Here is a proof that $\forall x \in \ints. \exists y \in \ints. (x(x+1) = 2y)$
using the following fact.\\[5pt]
{\bf Lemma} $\forall u \in \ints. \exists v \in \ints. [ (u = 2v) \Or (u= 2v-1)]$.

\noindent
After each semicolon (;) in the proof, justify why the line is true
(for example, state what proof technique is being used)
and  which earlier lines are being used, if any.

\begin{tabbing}
XXX\=XX\=XX\=XX\=XX\=XX\=XX\=\kill
 1.\>\>Let $x \in \ints$ be arbitrary.\\[5pt]
 2.\>\>$\exists v \in \ints. [ (x = 2v) \Or (x= 2v+1)]$;\\[5pt]
 3.\>\>\>Let $v$ be such that $(x = 2v) \Or (x= 2v-1)$;\\[5pt]
 4.\>\>\>\> Assume  $x = 2v$.\\[5pt]
 5.\>\>\>\>$x(x+1) = 2v(x+1)$; \\[5pt]
 6.\>\>\>\>\>Let $y =v(x+1)$.\\[5pt]
 7.\>\>\>\>\>$x(x+1) = 2y$; \\[5pt]
 8.\>\>\>\>$\exists y \in \ints. (x(x+1) = 2y)$; \\[5pt]
 9. \>\>\>$(x = 2v) \Implies \exists y \in \ints. (x(x+1) = 2y)$; \\[5pt]
10.\>\>\>\>Assume  $x = 2v-1$.\\[5pt]
11.\>\>\>\>$x+1 = 2v$; \\[5pt]
12.\>\>\>\>$x(x+1) = x2v$;\\[5pt]
13.\>\>\>\>$x2v = 2xv$; \\[5pt]
14.\>\>\>\>$x(x+1) = 2xv$; \\[5pt]
15.\>\>\>\>\>Let $y = xv$.\\[5pt]
16.\>\>\>\>\>$x(x+1) = 2y$; \\[5pt]
17.\>\>\>\>$\exists y \in \ints. (x(x+1) = 2y)$; \\[5pt]
18.\>\>\>$(x = 2v-1) \Implies \exists y \in \ints. (x(x+1) = 2y)$; \\[5pt]
19.\>\>\>$\exists y \in \ints. (x(x+1) = 2y)$; \\[5pt]
20.\>$\forall x \in \ints. \exists y \in \ints. (x(x+1) = 2y)$;
\end{tabbing}
\end{document}