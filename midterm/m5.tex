\documentclass[11pt]{article}
\usepackage{amssymb}
\usepackage{amsmath}
\usepackage{mathrsfs}
\usepackage{fullpage}
\usepackage{comment}
\includecomment{solution}
\includecomment{question}
\setlength{\parskip}{1ex}
\setlength{\parindent}{0in}

\newcommand{\Implies}{\mbox{ IMPLIES }}
\newcommand{\Or}{\mbox{ OR }}
\renewcommand{\And}{\mbox{ AND }}
\newcommand{\Not}{\mbox{NOT }}
\newcommand{\Iff}{\mbox{ IFF }}
\newcommand{\True}{\mbox{T}}
\newcommand{\False}{\mbox{F}}
\def\nats{{\mathbb N}}
\def\ints{{\mathbb Z}}
\def\reals{{\mathbb R}}
\def\setG{{\mathscr{G}}}

\begin{document}
\begin{center}
{\bf \Large \bf CSC240 Winter 2022 Midterm Examination Question 5}\\
\end{center}
\begin{solution}
    Sebastien Psarianos 1008596119
\end{solution}

\begin{question}
Your solution must be at most 1 typeset page, using at least 11 point font with 1 inch margins.
\end{question}


\begin{enumerate}
\setcounter{enumi}{4}
\item
\begin{question}
(8 marks)
Let  $\setG = \{ g:\nats \rightarrow \nats \ |\ \forall n \in \nats.(g(n+1) > 2{g(n)})\}$.\\
Use a diagonalization argument to prove that $\setG$ is uncountable.
\end{question}

\begin{solution}
    Assume on the contrary that $\setG$ is countable.\\
    There must exist a surjective function $h: \nats\rightarrow \setG$\\\\
    Therefore $g\in \setG$ can be defined as $g_i$ for some natural number $i$.\\
    The following matrix $M$ is defined such that $M[i,j]$ is the output of the function $g_j\in\setG$ with input $i$ (ie $g_3(2) = 15$) \\
    $$
        \left[
            \begin{matrix}
                0 & 1 & 3 & 7 & 15 & 31 & ...\\
                0 & 2 & 5 & 11 & 23 & 47 & ... \\
                0 & 3 & 7 & 15 & 31 & 63 & ...\\
                0 & 4 & 9 & 19 & 39 & 79 & ...\\
                0 & 5 & 11 & 23 & 47 & 95 & ...\\
                0 & 6 & 13 & 27 & 55 & 111 & ...\\
                ... & ... & ... & ...& ...& ... & ...
            \end{matrix}
        \right]
    $$
    Define $g_d: \nats\rightarrow \nats$ as $ \forall n\in\nats. (g(n) = M[n,n])$ (the diagonal of the matrix)\\
    Let $g_c$ be the function such that
    $$g_c(x) =\begin{cases}
        g_d(x) + 1 & 2g_d(x-1) +1  = g_d(x)\\
        2g_d(x-1) + 1 & 2g_d(x-1) + 1 \ne g_d(x)  \\
    \end{cases}$$  \\
    Since the value is always at least $1 + 2\times$ the previous value, this is a valid function in $\setG$. Therefore this set must be in the matrix. However this function has at least one different input/output combo from every other function in the matrix. This is a contradiction, therefore $\setG$ is not countable.
\end{solution}
\end{enumerate}
\end{document}