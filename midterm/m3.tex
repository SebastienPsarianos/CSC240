\documentclass[11pt]{article}
\usepackage{amssymb}
\usepackage{mathrsfs}
\usepackage{fullpage}
\usepackage{comment}
\includecomment{solution}
\excludecomment{question}
\setlength{\parskip}{1ex}
\setlength{\parindent}{0in}

\newcommand{\Implies}{\mbox{ IMPLIES }}
\newcommand{\Or}{\mbox{ OR }}
\newcommand{\And}{\mbox{ AND }}
\newcommand{\Not}{\mbox{NOT }}
\newcommand{\Iff}{\mbox{ IFF }}
\newcommand{\True}{\mbox{T}}
\newcommand{\False}{\mbox{F}}
\def\nats{{\mathbb N}}
\def\ints{{\mathbb Z}}
\def\reals{{\mathbb R}}

\begin{document}
\begin{center}
{\bf \Large \bf CSC240 Winter 2022 Midterm Examination Question 3}\\
\begin{solution}
Sebastien Psarianos 1008596119
\end{solution}
\end{center}

\begin{question}
Your solution must be at most 1 typeset page, using at least 11 point font with 1 inch margins.
\end{question}


\begin{enumerate}
\setcounter{enumi}{2}
\item
\begin{question}
(10 marks)
Consider the following recursively defined set ${\cal I}$ of propositional formulas:\\
Base Case: all propositional variables are in ${\cal I}$.\\
Constructor Case: If $E \in {\cal I}$ and $X$ is a propositional variable that does not occur in $E$, then $(E \Implies X) \in {\cal I}$.

Use structural induction to prove that no propositional formula in ${\cal I}$ is valid.\\\\
\end{question}
{\bf Definitions:}\\
${\cal V} = $ The set of all propositional variables. \\
$ {\rm Interp}(P, B): {\cal I} \times \{T,F\} \rightarrow \{T,F\} = $ There exists an interpretation in which the propositional formula $P$ has the value $B$.\\
$Q(P): {\cal I} \rightarrow \{T,F\} = {\rm Interp}(P, T) \And {\rm Interp}(P, F)$\\\\
{\bf Base Case:}\\
\null\quad Let $P\in {\cal V}$ be an arbitrary single propositional variable.\\
\null\quad $P\in {\cal I }$: (by the base case of ${\cal I}$) \\
\null\quad The interpretation $P = F$ makes the propositional formula $F$ \\
\null\quad ${\rm Interp}(P, F)$\\
\null\quad The interpretation $P = T$ makes the propositional formula $T$ \\
\null\quad ${\rm Interp}(P, T)$\\
\null\quad ${\rm Interp}(P, T) \And {\rm Interp}(P, F)$ (proof of conjunction)\\
\null\quad $Q(P)$\\
Since $P\in {\cal V}$ is arbitrary: $\forall P\in {\cal V}.  Q(P)$\\\\
{\bf Constructor Cases:}\\
\null\quad Let $P\in {\cal I}$ be arbitrary.\\
\null\qquad Assume $Q(P)$\\
\null\qquad ${\rm Interp}(P, T) \And {\rm Interp}(P, F)$ (by definition of $Q(P)$)\\
\null\qquad\quad Let $X\in{\cal V}$ be arbitrary.\\
\null\qquad\quad By the constructor case of ${\cal I}$, $(P\Implies X)\in {\cal I}$\\
\null\qquad\quad ${\rm Interp}(P, F)$ (use of conjunction)\\
\null\qquad\quad Keeping all values in the interpretation that makes $P = T$ the same, and assigining a\\
\null\qquad\quad value of $X= T$ gives a new interpretation that includes $X$.\\
\null\qquad\quad This interpretation will be $T$ since $T\Implies T$ is $T$\\
\null\qquad\quad ${\rm Interp}(P\Implies X, T)$\\
\null\qquad\quad ${\rm Interp}(P, T)$ (use of conjunction)\\
\null\qquad\quad Keeping all values in the interpretation that makes $P = T$ the same, and assigining a\\
\null\qquad\quad value of $X= F$ gives a new interpretation that includes $X$.\\
\null\qquad\quad This interpretation will be $F$ since $T\Implies F$ is $F$\\
\null\qquad\quad ${\rm Interp}(P\Implies X, F)$\\
\null\qquad\quad ${\rm Interp}(P\Implies X, T)\And {\rm Interp}(P\Implies X, F) $ (proof of conjunction)\\
\null\qquad\quad $Q(P\Implies X)$\\
\null\qquad $\forall X\in{\cal V}.Q(P\Implies X)$ (generalization)\\
\null\quad $Q(P)\Implies \forall X\in{\cal V}.Q(P\Implies X)$ (proof of implication)\\
$\forall P\in {\cal I}.\left[Q(P)\Implies \forall X\in{\cal V}.Q(P\Implies X)\right]$ (generalization)\\
$\forall P\in {\cal I}.Q(P)$ (structural induction)\\\\
Since every propositional formula in ${\cal I}$ can be both $T$ and $F$ with different interpretations, no formula in ${\cal I}$ is valid.
\begin{solution}
\end{solution}
\end{enumerate}
\end{document}