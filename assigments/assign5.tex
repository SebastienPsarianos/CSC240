\documentclass[11pt]{article}
\usepackage{amssymb}
\usepackage{fullpage}
\usepackage{mathrsfs}
\usepackage{comment}
\excludecomment{ignore}
\includecomment{solution}
\includecomment{question}
\setlength{\parindent}{0ex}
\setlength{\parskip}{1ex}
\def\nats {{\mathbb N}}
\def\ints {{\mathbb Z}}
\newcommand{\Implies}{\mbox{ IMPLIES }}
\newcommand{\Or}{\mbox{ OR }}
\newcommand{\And}{\mbox{ AND }}
\newcommand{\Not}{\mbox{NOT }}
\newcommand{\Iff}{\mbox{ IFF }}
\newcommand{\True}{\mbox{T}}
\newcommand{\False}{\mbox{F}}
\newcommand{\Subsets}[1]{\mathscr{P}_{#1}(\{1,\ldots N\})}

\begin{document}
\begin{center}
Solutions to\\
{\bf \Large \bf CSC240 Winter 2022 Homework Assignment 5}
\end{center}

\noindent
{\bf My name and student number:}\\
Sebastien Psarianos 1008596119
\medskip

\noindent
{\bf The list of people with whom I discussed this homework assignment:}\\
NO OUTSIDE DISCUSSION

\begin{question}
Consider the following three recursively defined sets of strings of $p$'s and $q$'s:\\
 \\
Base Case: $\lambda \in R$\\
Constructor Case: if $x,y \in R$, then $pxqy, qxpy \in R$.\\
 \\
Base Case: $\lambda \in S$\\
Constructor Case: if $x,y \in S$, then $pxq, qxp, xy \in S$.\\
 \\
Base Case: $\lambda \in U$\\
Constructor Case: if $x \in U$, then $pqx, qpx, pxq, qxp, xpq, xqp \in U$.\\
 \\
Let $T$ denote the set of all strings $w \in \{p,q\}^*$ of $p$'s and $q$'s containing the same number
of $p$'s and $q$'s.
\end{question}

\begin{enumerate}
\item
\begin{question}
Use structural induction to prove that $R \subseteq S$.
\end{question}

\begin{solution}
{\bf Solution:}\\\\
{\bf  Proof by structural induction}.\\\\
{\bf Definitions: }\\
$P(n): R\rightarrow\{T,F\} = $ $n$ is an element in $S$\\\\
{\bf Base Case:}\\
\null\quad Assume $n=\lambda$\\
\null\quad $\lambda$ is the base case in the definition of $S$, $\lambda\in S$\\
\null\quad $P(\lambda)$\\\\
{\bf Constructor Cases}:\\
\null\quad Let $w,z\in R$ be arbitrary\\\\
\null\qquad Assume $P(w)\And P(z)$ (I.H)\\
\null\qquad By the I.H, $w,z\in S$\\
\null\qquad By $S$'s constructor case $pxq$, $p\cdot w\cdot q\in S$.\\
\null\qquad By $S$'s constructor case $xy$ and since $pwq\in S$, $pwq\cdot z\in S$.\\
\null\qquad Therfore $P(pwqz)$\\
\null\quad Therefore $[P(w)\And P(z)]\Implies P(pwqz)$\\\\
\null\qquad Assume $P(w)\And P(z)$ (I.H)\\
\null\qquad By the I.H, $w,z\in S$\\
\null\qquad By $S$'s constructor case $qxp$, $q\cdot w\cdot p\in S$.\\
\null\qquad By $S$'s constructor case $xy$ and since $qwp\in S$, $qwp\cdot z\in S$.\\
\null\qquad Therfore $P(qwpz)$\\
\null\quad Therefore $[P(w)\And P(z)]\Implies P(qwpz)$\\\\
By structural induction, $\forall n\in R. P(n)$\\\\

\end{solution}

\item
\begin{question}
Use structural induction to prove that $S \subseteq T$.
\end{question}

\begin{solution}
{\bf Solution:}\\
{\bf  Proof by structural induction}.\\\\
{\bf Definitions:}\\
$N_q(n) \{p,q\}^* \rightarrow \nats = $ Number of occurances of $q$ in $n$\\
$N_p(n) \{p,q\}^* \rightarrow \nats = $ Number of occurances of $p$ in $n$\\
$P(n): \{p,q\}^* \rightarrow \{T,F\}= \left[N_q(n) = N_p(n) \right]$\\\\
{\bf Base Case:}\\
\null\quad Assume $n=\lambda$\\
\null\quad $\lambda$ contains $0$ $p$ and $0$ $q$ \\
\null\quad $P(\lambda)$\\\\
{\bf Constructor Cases:}\\
\null\quad Let $x,y\in S$ be arbitrary\\\\
\null\qquad Assume $P(x)$ (I.H)\\
\null\qquad By the I.H: $N_q(x) = N_p(x)$\\
\null\qquad $N_q(pxq) = N_q(x) + 1=N_p(x) + 1= N_p(pxq)$\\
\null\qquad $N_q(pxq) = N_p(pxq)$\\
\null\quad Therefore $P(x)\Implies P(pxq)$\\\\
\null\qquad Assume $P(x)$ (I.H)\\
\null\qquad By the I.H: $N_q(x) = N_p(x)$ \\
\null\qquad $N_q(qxp) = N_q(x) + 1=N_p(x) + 1= N_p(qxp)$\\
\null\qquad $N_q(qxp) = N_p(qxp)$\\
\null\quad Therefore $P(x)\Implies P(pxq)$\\\\
\null\qquad Assume $P(x)\And P(y)$ (I.H)\\
\null\qquad $N_q(xy) = N_q(x) + N_q(y)$\\
\null\qquad $N_p(xy) = N_p(x) + N_p(y)$\\
\null\qquad By the I.H: $\left[N_q(x) = N_p(x)\right]\And\left[N_q(y) = N_p(y)\right]$\\
\null\qquad Therefore $N_q(xy) = N_p(xy)$\\
\null\quad Therefore $(P(x)\And P(y))\Implies P(pxq)$\\\\
By structural induction $\forall n\in S.P(n)$\\
Every element in $S$ has equal occurances of $p$ and $q$, and by extension is in the set $T$
\end{solution}

\item
\begin{question}
Use strong induction to prove that $T \subseteq R$.
\end{question}

\begin{solution}
{\bf Solution:}\\
{\bf Proof by Strong Induction: }\\\\
{\bf Definitions:}\\
$Q(z): T\rightarrow\{T,F\} = $ $z$ is an element in $R$\\\\
Since $T$ has equal occurances of $p$ and $q$, $\forall b\in T.\exists m\in \nats. |b| = 2m$ where $m$ is the amount of $p$s or $q$s\\
To prove $\forall n\in \nats.Q(n)$ for each specific string length, define $P(n)$ as:\\
$P(n): \nats \rightarrow \{T,F\} = \forall z\in T.\left[(|z| = 2n) \Implies Q(z) \right]$\\\\
{\bf Base Cases: }\\
Case 1: $n=0$ \\
\null\quad Let $b\in T$ be arbitrary.\\
\null\qquad Assume that $|b| = 0$ \\
\null\qquad $b$ must be the empty string $\lambda$\\
\null\qquad By the base case of $R$, $Q(\lambda)$\\
\null\quad Therefore $|b|=0\Implies Q(b)$\\
Since $b$ is arbitrary $\forall b\in T.\left[(|b|=2(0))\Implies Q(b)\right]$\\
$P(0)$\\\\
Case 2: $n=1$ \\
\null\quad Let $b\in T$ be arbitrary.\\
\null\qquad Assume that $|b|=2(1)$\\
\null\qquad There are only two strings with 1 $p$ and 1 $q$: $pq$ and $qp$\\
\null\qquad By the base case of $R$, $\lambda\in R$ \\
\null\qquad By constructors $pxqy$ and $qxpy$, choose $x,y=\lambda$ giving $pq,qp\in R$ \\
\null\qquad $Q(pq)\And Q(qp)$\\
\null\quad Since $|b|=2(1)$ was assumed, $|b|=2(1)\Implies Q(b)$\\
Since $b$ is arbitrary $\forall b\in T.\left[(|b|=2(1))\Implies Q(b)\right]$\\
$P(1)$\\\\
{\bf Induction Step: }\\
Let $n\in \nats$ be arbitrary\\
\null\quad Assume $\forall k\in \nats,\left(k<n \Implies P(k)\right)$ (I.H)\\
\null\quad Let $b\in T$ be artbitrary \\
\null\qquad Assume $|b| = 2n$ \\
\null\qquad Since the base case for $R$ is $\lambda$, the strings $pq,qp\in R$ by constructor cases $pxqy$ and $qxpy$ \\
\null\qquad when $x,y = \lambda$\\
\null\qquad By the constructor case $\forall x,y\in R.Q(xy)$\\
\null\qquad Since $n-1<n$, by the I.H. $\forall x\in T.\left[|x|=2(n-1)\Implies Q(b)\right]$. \\


By strong induction: $\forall n\in \nats. P(n)$\\
\end{solution}

\item
\begin{question}
Prove that $U \neq R$.
\end{question}

\begin{solution}
{\bf Solution:}\\

\end{solution}
\end{enumerate}
\end{document}