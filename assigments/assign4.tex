\documentclass[11pt]{article}
\usepackage{amssymb}
\usepackage{fullpage}
\usepackage{mathrsfs}
\usepackage{comment}
\excludecomment{ignore}
\includecomment{solution}
\includecomment{question}
\setlength{\parindent}{0ex}
\setlength{\parskip}{1ex}
\def\nats {{\mathbb N}}
\def\ints {{\mathbb Z}}
\newcommand{\Implies}{\mbox{ IMPLIES }}
\newcommand{\Or}{\mbox{ OR }}
\newcommand{\And}{\mbox{ AND }}
\newcommand{\Not}{\mbox{NOT }}
\newcommand{\Iff}{\mbox{ IFF }}
\newcommand{\True}{\mbox{T}}
\newcommand{\False}{\mbox{F}}
\newcommand{\Subsets}[1]{\mathscr{P}_{#1}(\{1,\ldots N\})}

\begin{document}
\begin{center}
Solutions to\\
{\bf \Large \bf CSC240 Winter 2022 Homework Assignment 4}
\end{center}

\noindent
{\bf My name and student number:}\\

\medskip

\noindent
{\bf The list of people with whom I discussed this homework assignment:}\\


\begin{enumerate}
\item
\begin{question}
For any nonempty string of bits $b \in \{0,1\}^+$,
let $f(b)$ be the string of bits obtained by
inserting a 0 to the right of the leftmost bit in $b$ and then, if the string contains at least one 1,
changing the rightmost 1 to 0 and changing all the bits to its right to 1.\\
For example, $f(11) = 100$, $f(100)  = 0111$, and $f(0111) = 00110$.

Prove, using induction, that for all $b \in \{0,1\}^+$,  there exists $i \in \nats$ such that $f^{(i)}(b) \in \{0\}^+$
 is a nonempty string containing only 0's.
In other words, if you repeatedly apply the function $f$, eventually you will get a nonempty string containing only 0's.
Note that $f^{(0)}(b) = b$ for all nonempty strings $b \in \{0,1\}^+$.
\end{question}

\begin{solution}
{\bf Solution:}\\

\end{solution}


\item
\begin{question}
In one turn, a knight on a chessboard can either move two squares horizontally and one square vertically
or move two squares vertically and one square horizontally.
\end{question}

\begin{enumerate}
\item
\begin{question}
Using induction, prove that, for all $n \geq 3$, starting from every square on the border of an $n \times n$ chessboard, there is a sequence of turns that allows the knight to reach the square in the lower left corner.
\end{question}

\begin{solution}
{\bf Solution:}\\
\end{solution}

\item
\begin{question}
For $n \in \{1,2\}$, is it true that, starting from every square on the border of an $n \times n$ chessboard, there is a sequence of turns
that allows the knight to reach the square in the lower left corner? Justify your answer.
\end{question}

\begin{solution}
{\bf Solution:}\\

\end{solution}
\end{enumerate}
\end{enumerate}
\end{document}
