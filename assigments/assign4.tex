\documentclass[11pt]{article}
\usepackage{amssymb}
\usepackage{fullpage}
\usepackage{mathrsfs}
\usepackage{comment}
\excludecomment{ignore}
\includecomment{solution}
\includecomment{question}
\setlength{\parindent}{0ex}
\setlength{\parskip}{1ex}
\def\nats {{\mathbb N}}
\def\ints {{\mathbb Z}}
\newcommand{\Implies}{\mbox{ IMPLIES }}
\newcommand{\Or}{\mbox{ OR }}
\newcommand{\And}{\mbox{ AND }}
\newcommand{\Not}{\mbox{NOT }}
\newcommand{\Iff}{\mbox{ IFF }}
\newcommand{\True}{\mbox{T}}
\newcommand{\False}{\mbox{F}}
\newcommand{\Subsets}[1]{\mathscr{P}_{#1}(\{1,\ldots N\})}

\begin{document}
\begin{center}
Solutions to\\
{\bf \Large \bf CSC240 Winter 2022 Homework Assignment 4}
\end{center}

\noindent
{\bf My name and student number:}\\
Sebastien Psarianos 1008596119

\medskip

\noindent
{\bf The list of people with whom I discussed this homework assignment:}\\
NO OUTSIDE DISCUSSION

\begin{enumerate}
\item
\begin{question}
For any nonempty string of bits $b \in \{0,1\}^+$,
let $f(b)$ be the string of bits obtained by
inserting a 0 to the right of the leftmost bit in $b$ and then, if the string contains at least one 1,
changing the rightmost 1 to 0 and changing all the bits to its right to 1.\\
For example, $f(11) = 100$, $f(100)  = 0111$, and $f(0111) = 00110$.

Prove, using induction, that for all $b \in \{0,1\}^+$,  there exists $i \in \nats$ such that $f^{(i)}(b) \in \{0\}^+$
 is a nonempty string containing only 0's.
In other words, if you repeatedly apply the function $f$, eventually you will get a nonempty string containing only 0's.
Note that $f^{(0)}(b) = b$ for all nonempty strings $b \in \{0,1\}^+$.
\end{question}

\begin{solution}
{\bf Solution:}\\
${\rm LENGTH}(b,n):\{0,1\}^+\times\nats\rightarrow \{T,F\}$ The string $b$ contains $n$ numbers\\\\
$P: \{\nats \rightarrow \{T,F\}\}$ For every binary string of length $n$, there exists a natural number $i$ so that it can be reduced to a string containing only $0$s by applying $f$ to it $i$ times.\\
$P(n)= \forall b\in\{0,1\}^+.\left[{\rm LENGTH}(b,n) \Implies \left(\exists i\in\nats.f^{(i)}(b)\in\{0\}^+\right)\right]$\\\\

Let $n\in\nats$ be arbitrary\\
\null\quad Assume that $\forall k\in\nats.\left[k<n \Implies P(k)\right]$ (Induction Hypothesis)\\\\
\null\quad Case 1: $n=0$\\
\null\qquad Let $b\in\{0,1\}^+$ be an arbitrary binary string.\\
\null\qquad\quad Assume ${\rm LENGTH}(b,0)$\\
\null\qquad\quad Choose $i=0$ such that: $f^0(b) = b$.\\
\null\qquad\quad Since $b$ is empty: $b\in\{0\}^+$.\\
\null\qquad\quad Therefore $\exists i\in \nats.f^{(i)}(b)\in\{0\}^+$\\
\null\qquad Therefore ${\rm LENGTH}(b,0) \Implies \left[\exists i\in \nats.f^{(i)}(b)\in\{0\}^+\right]$\\
\null\quad $\forall b\in \{0,1\}^+\left({\rm LENGTH}(b,0) \Implies \left[\exists i\in\nats.f^{(i)}(b)\in\{0\}^+\right]\right)$.\\
\null\quad Therefore $P(0)$
\\\\
\null\quad Case 2: $n>0$\\
\null\quad Since $n>0$ and by the I.H, $P(n-1)$\\
\null\qquad Let $b\in\{0,1\}^+$ be an arbitrary binary string.\\
\null\qquad\quad Assume ${\rm LENGTH}(b,n)$ \\
\null\qquad\quad Since $n>1$ let $b'$ be the string $b$ with the last bit removed.\\
\null\qquad\quad The string $b'$ has a length of $n-1$\\
\null\qquad\quad The first bit in $b$ is either a $0$ or a $1$.\\
\null\qquad\qquad Assume the first bit in $b$ is a $1$\\\\
\null\qquad\qquad Every time the $f$ is applied to the binary string two things happen:\\
\null\qquad\qquad A zero is added to the right of the first bit, increasing the value of the string by\\
\null\qquad\qquad $2^{z+1} - 2^z$ (where $z$ is the length of the string before $f$ was applied). \\
\null\qquad\qquad The rightmost $1$ is removed and subsequent bits become $1$s reducing the value of the\\
\null\qquad\qquad string by $x\ge1$ unless $b\in\{0\}^+$. \\
\null\qquad\qquad By the I.H $\exists i\in \nats.f(b')\in\{0\}^+$ meaning that eventually every bit that was a \\
\null\qquad\qquad part of $b'$ will be removed. When this happens, since only $0$s are added to the string,\\
\null\qquad\qquad the string will be a $1$ followed by only $0$s. \\
\null\qquad\qquad Applying the function again, the first bit will turn into a zero and every subsequent\\
\null\qquad\qquad bit will become a $1$. The numerical value of the string won't increase when applying\\
\null\qquad\qquad $f$ subsequent times. The string will always start with $0$s. \\
\null\qquad\qquad The step of the function that removes the last bit will continue to reduce the value\\
\null\qquad\qquad of the function every time $f$ is applied resulting in the value eventually reaching $0$.\\
\null\qquad\quad $b_0= 0 \Implies \exists i\in\nats.f^{(i)}(b)\in\{0\}^+$\\\\
\null\qquad\qquad Assume the first bit in $b$ is $0$\\
\null\qquad\qquad Considering the nuerical value of the binary string, if the string starts with a zero,\\
\null\qquad\qquad adding $0$s after the first bit will not affect the value of the string. \\
\null\qquad\qquad However the  removal of the rightmost bit and changing of the subsequent values to \\
\null\qquad\qquad $1$ will always reduce the value of the string by at least $1$. Since the value is decreasing\\
\null\qquad\qquad every time $f$ is applied, it will eventually reach $0$ and the string will only contain $0$s. \\
\null\qquad\quad $b_0= 0 \Implies \exists i\in\nats.f^{(i)}(b)\in\{0\}^+$\\\\
\null\qquad ${\rm LENGTH}(b,n)\Implies \exists i\in\nats.f^{(i)}(b)\in\{0\}^+$\\\\
\null\quad $\forall b \in\{0,1\}^+.{\rm LENGTH}(b,n)\Implies \exists i\in\nats.f^{(i)}(b)\in\{0\}^+$\\
\null\quad Therefore, $P(n)$ \\\\
By induction $\forall n\in\nats.P(n)$
\\\\

\end{solution}


\item
\begin{question}
In one turn, a knight on a chessboard can either move two squares horizontally and one square vertically
or move two squares vertically and one square horizontally.
\end{question}

\begin{enumerate}
\item
\begin{question}
Using induction, prove that, for all $n \geq 3$, starting from every square on the border of an $n \times n$ chessboard, there is a sequence of turns that allows the knight to reach the square in the lower left corner.
\end{question}

\begin{solution}
{\bf Solution:}\\
Let $p(n): \nats\rightarrow\{T,F\}=$ It is possible to get to the bottom left corner of a $n\times n$ chessboard starting at any point on the border.\\\\
Let $n\in\nats$ be arbitrary.\\
\null\quad Assume that $\forall k\in\nats.\left[3\le k<n\Implies P(k)\right]$\\\\
\null\qquad {\bf Case 1: $n = 3$}\\
\null\qquad Define an edge as one row or column on the board where every square is part of the\\
\null\qquad border. On a $3\times3$ board every edge has 3 squares (2 corners and 1 middle square)\\\\
\null\qquad For every corner, there is a unique pair of edges that don't contain that corner.\\
\null\qquad (opposing sides)\\
\null\qquad If the knight is on a corner, it can move to the middle square of either of\\
\null\qquad its opposing sides. (2 moves along an edge 1 towards the centre) \\\\
\null\qquad For every edge, there is a unique pair of corners that are not part of that edge.\\
\null\qquad (opposing corners)\\
\null\qquad If the knight is in the middle square of an edge, it can always move to the two \\
\null\qquad opposing corners (one move along its edge and two moves toward the corner).\\\\
\null\qquad Given two corners that are on the same edge ($a$ and $b$). Corner $a$ and $b$ will \\
\null\qquad always share one opposing edge. The middle square of the shared opposing edge  \\
\null\qquad can be reached from both $a$ and $b$, therefore you can always move from $a$ to $b$ by  \\
\null\qquad visiting the middle opposing square. This works in both directions. \\\\
\null\qquad By doing this repeatedly in the same direction, it is possible to reach every  \\
\null\qquad corner on the board from any other corner.\\
\null\qquad Since there are two opposing corners for every middle square on an edge, you can \\
\null\qquad always make it to a corner from a middle square of an edge.\\\\
\null\qquad Since for a $3\times 3$ board any square on the border is either a middle square or   \\
\null\qquad a corner, it is possible to get to any corner from any spot on the border and by \\
\null\qquad extension it is always possible to get to the bottom left corner.\\
\null\quad Therefore $P(3)$\\\\
\null\qquad {\bf Case 2: $n=4$}\\
\null\qquad A $4\times4$ board is a $3\times3$ board with one column added to the right and one row\\
\null\qquad added to the top.\\
\null\qquad On the left border every square except the top one is part of the $3\times 3$ board's\\
\null\qquad border, therefore by the I.H if the knight is placed on these squares then it is able \\
\null\qquad to reach the bottom left square.\\
\null\qquad The same holds for the bottom border and the rightmost square.\\

\null\qquad If the knight is placed on the top left corner then it can be moved down 1 and right\\
\null\qquad  2 to end up on the $3\times3$ board's border.\\
\null\qquad If the knight is placed to the right of the top left corner then it can be moved down \\
\null\qquad 2 and right 1 to end up on the $3\times3$ board's border.\\
\null\qquad If the knight is placed to the left of the top right corner or on the top right corner\\
\null\qquad  then it can be moved down 1 and left 2 to end up on the $3\times3$ board's border.\\
\null\qquad If the knight is placed below the top right corner, then it can be moved left 1 and \\
\null\qquad down 2 to end up on the $3\times3$ board's border.\\
\null\qquad If the knight is placed above the bottom right corner, then it can be moved left 2 \\
\null\qquad and down 1 to end up on the $3\times3$ board's border.\\
\null\qquad If the knight is placed on the bottom right corner then it can be moved left 1 and \\
\null\qquad up 2 to end up on the $3\times3$ board's border.\\
\null\qquad Therefore from any point on the border of the $4\times4$ board, the border of the $3\times3$\\
\null\qquad board can be reached and by the I.H, the bottom left square can be reached.\\
\null\quad Therefore $P(4)$\\\\
\null\qquad {\bf Case 3: $n > 3$}\\
\null\qquad The $n\times n$ board is a $(n-1)\times(n-1)$ board with a column and row added to the\\
\null\qquad right and top respectively.\\
\null\qquad Let the outer board refer to the $n\times n$ board and the inner board refer to the\\
\null\qquad $(n-1)\times (n-1)$ board.\\
\null\qquad In this configuration, the bottom left square of the inner board overlaps with the\\
\null\qquad bottom left square of the outer board.\\\\
\null\qquad Since the knight must be on the border, it will either be on the border of the\\
\null\qquad inner board or it will be on one of the two rows that aren't shared with the inner \\
\null\qquad board.\\\\
\null\qquad If the knight is on the border of the inner board, by the I.H it is possible to move it\\
\null\qquad to the bottom left square of the inner board which the same as the bottom left \\
\null\qquad square of the outer board.\\\\
\null\qquad If the knight is not on the border of the inner board, then it must be on either \\
\null\qquad the top row or the rightmost column.\\
\null\qquad From all positions on the top row move the knight down 1 and 2 right or left\\
\null\qquad depending its current column. (right works if there are $\ge 3$ columns to the \\
\null\qquad right or left works if there $\ge 2$ columns to the left)\\
\null\qquad This will always leave the knight on the border of the inner board. By the I.H it is\\
\null\qquad now possible to move to the bottom left corner.\\
\null\qquad The same holds for the right column but with one move to the left and the same\\
\null\qquad rules for moving up and down.\\
\null\quad Therefore $P(n).n>4$\\\\
$\forall n\in\nats.P(n)$, by strong induction\\\\

\end{solution}

\item
\begin{question}
For $n \in \{1,2\}$, is it true that, starting from every square on the border of an $n \times n$ chessboard, there is a sequence of turns
that allows the knight to reach the square in the lower left corner? Justify your answer.
\end{question}

\begin{solution}
{\bf Solution:}\\
It is possible on a $1\times1$ chessboard. The single square could be considered to be the bottom-left square. Therefore since there is only one square that the knight can be placed, it can always reach the bottom-left square with 0 moves.\\
Therefore $P(1)$\\\\
It is not possible on a $2\times2$ chessboard. This is because for the knight to move it must have at least 3 squares in a row. Therefore if the knight is placed on a square that is not the bottom left one, it is impossible for it to move to another square. \\
Therefore $\Not(P(2))$
\end{solution}
\end{enumerate}
\end{enumerate}
\end{document}
