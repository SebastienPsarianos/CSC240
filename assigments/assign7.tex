\documentclass[11pt]{article}
\usepackage{amssymb}
\usepackage{fullpage}
\usepackage{mathrsfs}
\usepackage{comment}
\excludecomment{ignore}
\includecomment{solution}
\excludecomment{question}
\includecomment{questiononly}
\setlength{\parindent}{0ex}
\setlength{\parskip}{1ex}
\def\nats {{\mathbb N}}
\def\ints {{\mathbb Z}}
\def\reals {{\mathbb R}}
\newcommand{\Implies}{\mbox{ IMPLIES }}
\newcommand{\Or}{\mbox{ OR }}
\newcommand{\And}{\mbox{ AND }}
\newcommand{\Not}{\mbox{NOT }}
\newcommand{\Iff}{\mbox{ IFF }}
\newcommand{\True}{\mbox{T}}
\newcommand{\False}{\mbox{F}}
\newcommand{\Subsets}[1]{\mathscr{P}_{#1}(\{1,\ldots N\})}
\usepackage{graphicx}

\begin{document}
\begin{center}
\begin{solution}
Solutions to\\
\end{solution}

{\bf \Large \bf CSC240 Winter 2022 Homework Assignment 7}\\

\begin{questiononly}
due Thursday March 17, 2022
\end{questiononly}

\end{center}
{\bf My name and student number:}\\
Sebastien Psarianos 1008596119\\
{\bf The list of people with whom I discussed this homework assignment:}\\
NO OUTSIDE DISCUSSION\\

\begin{question}
The following  algorithm sorts the elements $L[i..j]$ of the array $L$ in nondecreasing order.
It does so by recursively sorting the first two thirds of the array,
then sorting the last two thirds of the array, and then sorting
the first two thirds of the array again.

 \begin{tabbing}
XXXXXXXXXXXXXXXXXXX\=XX\=XX\=\+\kill
    SortByThirds$(L,i,j)$ \\
    1 \hspace{2mm} \= if \= $j-i \geq 2$ then \\
    2 \> \> $m \leftarrow \lfloor \frac{j-i+1}{3} \rfloor$  \\
    3 \> \> SortByThirds$(L,i, j - m)$\\
    4 \> \> SortByThirds$(L,i + m , j)$\\
    5 \> \> SortByThirds$(L,i, j - m)$\\
    6 \> else if $j-i=1$ then  \\
    7 \> \> if \= $L[j] < L[i]$ then  \\
    8 \> \> \> $x \leftarrow L[j]$ \\
    9 \> \> \> $L[j] \leftarrow L[i]$ \\
    10 \> \> \> $L[i] \leftarrow x$
  \end{tabbing}
\end{question}

\begin{enumerate}
\item
\label{recurrence}
\begin{question}
Write a recurrence describing the number of comparisons performed by SortByThirds($L,i,j$)
as a function of the size of the list being sorted. The comparisons are $<$, =, and $\geq$. \\
Justify your answer.
\end{question}

\begin{solution}
{\bf Solution}:\\
Given the base case of $j-i = 1$, the line $1$ comparison will fail and the line $6$ comparison will pass. Therefore the comparison on line $7$ will also run.
This means that given $n=j-i-=1$, $T(1) = 3$\\
Since both intervals $L[i..j-m]$ and $L[i+m..j]$ have a size of $j-i -m$ and since $m = \lfloor(j-i+1)/3\rfloor$ the size of the array that is being compared in each call to $SortByThirds$ is $j - i - m = j-i -\lfloor(j-i+1)/3\rfloor$ Since the floor of $(j-i+1)/3$ is being subtracted from $j-i$, this is equivalent to $\lceil j - i - (j-i+1)/3 \rceil = \lceil (3j-3i-j+i - 1)/3\rceil = \lceil (2j-2i - 1)/3 \rceil$\\
Since $n = j - i$, this is equivalent to $\lceil (2n-1)/3\rceil =  \lceil \frac{2n}3- \frac13\rceil$\\
Since every $SortByThirds$ recursive call is on an array of size $\lceil \frac{2n}3- \frac13\rceil$, $T(\lceil \frac{2n}3- \frac13\rceil)$ comparisons are made. Since there are three $SortByThirds$ calls and since there is one comparison at line $1$ that is always true, there is one additional comparison. Therefore $T(n) = 3T(\lceil \frac{2n}3 - \frac 13\rceil) + 1$ when $n\ge2$\\
$$T(n) \in \left\{
\begin{array}{ll}
3 & n < 2\\
3T(\lceil \frac{2n}3 - \frac13 \rceil)+ 1&  n\ge2
\end{array} \right .$$

\end{solution}

\item
\begin{question}
Use the following slight generalization of the Master Theorem
(with $b \in \reals^+$, rather than in $\ints^+$)
to solve your
recurrence in the previous question, to within a constant factor.
For all $n \in \ints^+$,
$$T(n) = \left\{
\begin{array}{ll}
c &  \mbox{if } n < B\\
a_1 T( \lceil n/b \rceil) + a_2 T( \lfloor n/b \rfloor) + dn^i & \mbox{if } n \geq B,
\end{array} \right .$$
where $a_1, a_2, B\in \nats$, $a = a_1 + a_2 \geq 1$, $b > 1$,  and $b,c,d, i \in \reals^+ \cup \{0\}$, then
$$T(n) \in \left\{
\begin{array}{ll}
\Theta(n^i \log n) &  \mbox{if } a= b^i\\
\Theta(n^i ) &  \mbox{if } a< b^i\\
\Theta(n^{\log_b a}) &  \mbox{if } a>  b^i.
\end{array} \right .$$
\end{question}

\begin{solution}
{\bf Solution}:\\

\end{solution}

\item
\begin{question}
Suppose that
the elements of $L[1..3^k]$ are distinct and SortByThirds($L,1,2\cdot3^{k-1}$) is performed.
If SortByThirds is correct,
where are the largest $3^{k-1}$ elements of $L[1..3^k]$ located?\\
Briefly justify your answer.
\end{question}

\begin{solution}
{\bf Solution}:\\
$2\cdot3^{k-1} = \frac23 3^k = 3^k - \frac13 3^k = 3^k - 3^{k-1}$ \\
Therefore the interval $L[1..2\cdot3^{k-1}]$ is the same as $L[1..2\cdot3^{k} - 3^{k-1}]$.\\
Let $A = 3^{k-1}$\\
Since $SortByThirds(L, 1, 2A)$ is correct, every element in $L[1..2\cdot3^{k} -A]$ will be sorted in nondecreasing order. \\
Therefore the interval $L[2\cdot3^{k} - 2A..2\cdot3^{k} - A]$ will contain the largest $A$ elements in $L[1..2\cdot3^{k} - A]$.\\
Since $L[3^k-A..3^k]$ is not sorted by $SortByThirds$, if any of the largest $A$ values in $L[1..3^k]$ are in this inteval, they will not be moved. Therefore no matter where the largest values are, they will always be in the interval $L[3^k-2A..3^k]$\\

\end{solution}


\item
\begin{question}
Give a formal definition of what ``SortByThirds is correct'' means.
\end{question}

\begin{solution}
{\bf Solution}:\\
For any $i, j\in \ints^+$ and for any totally ordered array $L$ if $L[i..j]$ is a valid interval in $L$, $SortByThirds(L, i, j)$ will eventually halt and $L[i..j]$ will be sorted in nondecreasing order where $L$'s multiset is unchanged.
\end{solution}

\item
\begin{question}
Prove that SortByThirds is correct.
You may assume the elements of $L$ are distinct.
\end{question}

\begin{solution}
{\bf Definitions:}\\
Let $A$ represent the set of all arrays of any totally ordered sets\\\\
Let $B(L, i, j) = : A\times \ints^+\times\ints^+ \rightarrow \{T,F\}  $ "The inteveral $L[i..j]$ is a valid interval in $L$\\\\
Let $Q(L, i, j): A\times \ints^+\times\ints^+ \rightarrow \{T,F\}  = $ When $SortByThirds(L, i, j)$ is performed it eventually halts; at which time $L[i..j]$ is sorted in nondecreasing order and $L$s multiset is unchanged\\\\
Let $P(n): \ints^+ \rightarrow \{T,F\} = \forall i,j\in\ints^+.\forall L\in A.\left([B(L, i, j)\And n=j-i]\Implies Q(L, i, j)\right)$.

{\bf Solution}:\\
Claim: $\forall n\in \ints^+.P(n)$\\
\null\quad Let $n\in \ints^+$ be arbitrary\\
\null\quad {\bf Base Case: $n=1$ }\\
\null\qquad Let $i, j\in Z^+$ be arbitrary\\
\null\qquad\quad Let $L\in A$ be an arbitrary array\\
\null\qquad\qquad Assume $B(L, i, j) \And  j - i= 1$\\
\null\qquad\qquad Consider $SortByThirds(L, i, j)$\\
\null\qquad\qquad The test on line $1$ fails\\
\null\qquad\qquad The test on line $6$ passes\\
\null\qquad\qquad There are two cases (assuming that every element in $L$ is distinct)\\
\null\qquad\qquad\quad Assume $L[j]$ is greater than $L[i]$\\
\null\qquad\qquad\quad The test on line $7$ fails and $L$ is not altered.\\
\null\qquad\qquad\quad Since $j>i$, and since $L[j]>L[i]$ $L[i]$ and $L[j]$ are in nondecreasing order.\\
\null\qquad\qquad\quad The list is not changed therefore the multiset of $L$ remains the same.\\
\null\qquad\qquad $L[j]>L[i]\Implies Q(L, i, j)$\\
\null\qquad\qquad\quad Assume $L[i]$ is greater than $L[j]$\\
\null\qquad\qquad\quad The test on line $7$ passes.\\
\null\qquad\qquad\quad The values of $L[i]$ and $L[j]$ will be swapped (Lines $8,9,10$)\\
\null\qquad\qquad\quad $L[i]$ and $L[j]$ are now in nondecreasing order\\
\null\qquad\qquad\quad The multiset of $L$ remains unchanged since nothing was added/removed, \\
\null\qquad\qquad\quad only positions were swapped\\
\null\qquad\qquad $L[j] < L[i]\Implies Q(L, i, j)$\\
\null\qquad\qquad Therefore for all cases $Q(L, i, j)$\\
\null\qquad\quad $B(L, i, j) \And  j - i= 1\Implies Q(L, i, j)$ (direct proof of implication)\\
\null\qquad $\forall L\in A.\left([B(L, i, j) \And  j - i= 1]\Implies Q(L, i, j)\right)$ (generalization)\\
\null\quad $\forall m,i,j\in\ints^+.\forall L\in A.\left([B(L, i, j) \And  j - i= 1]\Implies Q(L, i, j)\right)$ (generalization)\\
\null\quad $P(1)$ (by definition)\\\\
\null\quad {\bf Induction Step:}\\
\null\qquad Assume $\forall n'\in\ints^+.n'<n \Implies P(n')$ (I.H)\\
\null\qquad\quad Let $i, j\in Z^+$ be arbitrary\\
\null\qquad\qquad Let $L\in A$ be an arbitrary array\\
\null\qquad\qquad\quad Assume $B(L, i, j) \And  j - i= 1$\\
\null\qquad\qquad\quad Consider $SortByThirds(L, i, j)$\\
\null\qquad\qquad\quad The test on line $1$ will pass since $n$ is at least $2$\\
\null\qquad\qquad\quad $m = \left\lfloor\frac{j-i+1}3\right\rfloor$ from line $2$\\
\null\qquad\qquad\quad Since $m$ is a third of the size of the interval floored, the array can be divided into \\
\null\qquad\qquad\quad thirds.\\
\null\qquad\qquad\quad $L[i..i+m]$ and $L[j-m..j]$ (the first and last thirds) both have size $(i+m) - i=m$ \\
\null\qquad\qquad\quad and $j-(j-m)=m$.\\\\
\null\qquad\qquad\quad The middle third $L[i+m..j-m]$ has a size of $j-m - (i+m) = j-i-2m$.\\
\null\qquad\qquad\quad Since $m$ is floored, subtracting $2m$ from the size of $L[i..j]$ will result in an interval \\
\null\qquad\qquad\quad that is greater than or equal to $m$\\\\
\null\qquad\qquad\quad By the I.H and by line $3$, $L[i..j-m]$ (first two thirds of $L$) is then sorted in \\
\null\qquad\qquad\quad nondecreasing order.\\
\null\qquad\qquad\quad By extension, $L[i+m..j-m]$ contains at least the $m$ largest items in $L[i..j-m]$\\\\
\null\qquad\qquad\quad By the I.H and by line $4$, $L[i+m..j]$ (last two thirds of $L$) is then sorted in \\
\null\qquad\qquad\quad nondecreasing order.\\
\null\qquad\qquad\quad Since $L[i+j..j-m]$ previously contained the $m$ largest items in $l[i..j-m]$ and is\\
\null\qquad\qquad\quad within $L[i+m..j]$, the interval $L[j-m..j]$ will now contain the largest $m$ items in \\
\null\qquad\qquad\quad $L[i..j]$ sorted in nondecreasing order.\\\\
\null\qquad\qquad\quad By the I.H and by line $5$, $L[i..j-m]$ is then sorted in nondecreasing order.\\
\null\qquad\qquad\quad Since the largest items are sorted at the last third of the array, when the rest of \\
\null\qquad\qquad\quad the items are sorted, every element will be in nondecreasing order on the interval \\
\null\qquad\qquad\quad $L[i..j]$\\
\null\qquad\qquad\quad By the I.H. none of the recursive calls to $SortByThirds$ will add or remove any \\
\null\qquad\qquad\quad items, therefore the multiset of the items in $L$ remains the same.\\
\null\qquad\qquad\quad $Q(L, i, j)$\\
\null\qquad\qquad $B(L, i, j) \And  j - i= n\Implies Q(L, i, j)$ (direct proof of implication)\\
\null\qquad\quad $\forall L\in A.\left([B(L, i, j) \And  j - i= n]\Implies Q(L, i, j)\right)$ (generalization)\\
\null\qquad $\forall m,i,j\in\ints^+.\forall L\in A.\left([B(L, i, j) \And  j - i= n]\Implies Q(L, i, j)\right)$ (generalization)\\
\null\qquad $P(n)$ (by definition)\\
\null\quad $(\forall n'\in\ints^+.n'<n \Implies P(n'))\Implies P(n)$ (direct proof of implication)\\
$\forall n\in\ints^+.(\forall n'\in\ints^+.n'<n \Implies P(n'))\Implies P(n)$ (generalization)\\
By strong induction $\forall n\in\ints^+.P(n)$\\\\
For any $i, j\in \ints^+$ and for any totally ordered array $L$ if $L[i..j]$ is a valid interval in $L$, $SortByThirds(L, i, j)$ will eventually halt and $L[i..j]$ will be sorted in nondecreasing order where $L$'s multiset is unchanged.
\end{solution}
\end{enumerate}
\end{document}

