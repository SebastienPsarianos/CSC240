\documentclass[11pt]{article}
\usepackage{amssymb}
\usepackage{fullpage}
\usepackage{mathrsfs}
\usepackage{comment}
\excludecomment{ignore}
\includecomment{solution}
\excludecomment{question}
\setlength{\parindent}{0ex}
\setlength{\parskip}{1ex}
\def\nats {{\mathbb N}}
\def\ints {{\mathbb Z}}
\newcommand{\Implies}{\mbox{ IMPLIES }}
\newcommand{\Or}{\mbox{ OR }}
\newcommand{\And}{\mbox{ AND }}
\newcommand{\Not}{\mbox{NOT }}
\newcommand{\Iff}{\mbox{ IFF }}
\newcommand{\True}{\mbox{T}}
\newcommand{\False}{\mbox{F}}
\newcommand{\Subsets}[1]{\mathscr{P}_{#1}(\{1,\ldots N\})}

\begin{document}
\begin{center}
Solutions to\\
{\bf \Large \bf CSC240 Winter 2022 Homework Assignment 6}
\end{center}

\noindent
{\bf My name and student number:}\\
Sebastien Psarianos 1008596119

\medskip

\noindent
{\bf The list of people with whom I discussed this homework assignment:}\\
NO OUTSIDE DISCUSSION


\begin{enumerate}
\item
\begin{question}
Consider the following  recursively defined set of numbers:\\
 \\
Base Case: $2 \in H$\\
Constructor Case: if $h \in H$, then $2+\sqrt{h}  \in H$ and $2-\sqrt{h} \in H$.\\
 \\
Prove, using a well ordering proof, that this definition is unambiguous,\\
i.e.~every number in $H$ can be obtained from the rules in only one way.\\
 \\
You may find it helpful to first prove upper and lower bounds on the numbers in $H$.
\end{question}

\begin{solution}
{\bf Solution:}\\
{\bf Lemma \#1:} Show $2$ is the only rational number in $H$ - Proof by structural induction:\\
{\bf Definitions}:\\
$F(h): H \rightarrow \{T,F\} = h\ne 2\Implies I(h)$\\
$I(x): \mathbb{R} \rightarrow \{T,F\} = x$ is a irrational number\\\\
{\bf Base Cases: }\\
\null\quad Assume $h=2$\\
\null\quad $Q(2)$\\
\null\quad Since $2\ne2$ is false, $2\ne 2\Implies I(2)$ is vaccuously true.\\
$F(2)$\\\\
{\bf Constructor Cases: }\\
\null\quad Let $h\in H$ be arbitrary:\\
\null\qquad Assume $F(h)$ (I.H.)\\
\null\qquad Case 1: $h = 2$\\
\null\qquad By the I.H. $h\ne 2\Implies I(h)$\\
\null\qquad Since $h=2$, $\Not I(h)$.\\
\null\qquad $I(\sqrt 2)$: Axiom \\
\null\qquad Addition/subtraction of a rational and irrational number is always irrational: Axiom\\
\null\qquad Therefore $I(2\pm\sqrt2)$\\
\null\qquad Case 2: $h\ne2$\\
\null\qquad By the I.H. $h\ne 2\Implies I(h)$\\
\null\qquad Since $h\ne2$, $I(h)$\\
\null\qquad The square root of an irrational number is also irrational: Axiom\\
\null\qquad Therefore $I(\sqrt h)$\\
\null\qquad $\Not I(2)$: Axiom\\
\null\qquad Addition/subtraction of a rational and irrational number is always irrational: Axiom:\\
\null\qquad $I(2\pm\sqrt h)$\\
\null\quad Therefore $F(h)\Implies F(2\pm\sqrt h)$\\
Since $h$ is arbitrary, by structural induction $\forall h\in H.F(h)$\\
Since every either constructor gives an irrational number no matter the input, there is only one way to obtain $2$. (The base case)\\\\
{\bf Lemma \# 2:} - Proving that $\forall.h\in H.(0\le h\le 4)$\\

{\bf Definitions:}\\
$P(x): H\rightarrow \{T,F\} = $ The element $x$ can only be obtained from the rules in one way\\
$x\preccurlyeq y: H\times H \rightarrow \{T,F\} =$ The number of times that a constructor was applied to make $x$ is less or equal to the amount for $y$ \\
{\bf Proof:}\\
Let $C =\{h\in H|\Not [P(h)]\}$\\
$\preccurlyeq$ is a well ordering for $H$ with $2$ as the smallest element. Since $C\subseteq H$ $C$ is well ordered.\\
\null\quad Assume on the contrary $C\ne\emptyset$ \\
\null\quad Let $x$ be the smallest element in $C$ by $\preccurlyeq$\\
\null\quad\quad Let $a\in H$ be arbitrary\\
\null\qquad\quad Assume $a\preccurlyeq x$\\
\null\qquad\quad As shown in the lemma \#1, the base case 2 can only be made in one way. The base \\
\null\qquad\quad case is by definition less than $x$ with the well ordering $\preccurlyeq$. \\
\null\qquad\quad Therefore there always exists a value $a\preccurlyeq x$ st $P(a)$\\
\null\quad\quad $a\preccurlyeq x\Implies P(a)$\\
\null\quad $\forall a\in H.(a\preccurlyeq x\Implies P(a))$\\
\null\quad Since $x\in C$, it must be able to be made in two ways. There are three possibilities:  \\
\null\qquad Case 1: $\exists a,b\in H.(2+\sqrt a = 2+\sqrt b = x)$\\
\null\qquad $2+\sqrt a = 2+\sqrt b$\\
\null\qquad $\sqrt a = \sqrt b$\\
\null\qquad Since all values in $H$ are greater than $0$ as shown in lemma 2:\\
\null\qquad $a = b$\\
\null\qquad $a\preccurlyeq x$ and $b\preccurlyeq x$ by the definition of $\preccurlyeq$, therefore $P(a)$ and $P(b)$\\
\null\qquad $a$ and $b$ are the same element obtained in the same way.\\
\null\qquad $ 2+\sqrt a$ and $2+\sqrt b$ are the same constructor\\
\null\qquad Case 2: $\exists a,b\in H.(2-\sqrt a = 2-\sqrt b = x)$ $a$ and $b$ obtained from the rules differently\\
\null\qquad $-\sqrt a = -\sqrt b$\\
\null\qquad Since all values in $H$ are greater than $0$ as shown in lemma 2:\\
\null\qquad $a = b$\\
\null\qquad $a\preccurlyeq x$ and $b\preccurlyeq x$ by the definition of $\preccurlyeq$, therefore $P(a)$ and $P(b)$\\
\null\qquad This is a contradiction since $a$ and $b$ must be obtained differently\\
\null\qquad Case 3: $\exists a,b\in H.(2+\sqrt a = 2-\sqrt b = x)$\\
\null\qquad $2+\sqrt a = 2-\sqrt b$ \\
\null\qquad $\sqrt a = -\sqrt b$\\
\null\qquad This is a contradiction since the output of a square root is always positive: Axiom\\
\null\quad Therefore there is no way to create the smallest element in $C$\\
By the well ordering principle: $C=\emptyset$\\
$\forall h\in H.P(h)$
\end{solution}

\item
\begin{question}
The set $S$ of strict binary trees can be defined inductively as follows:\\
Base case: $\bullet$ is a strict binary tree, with $\bullet$ as its root\\
Constructor case: If $t_1$ and $t_2$ are strict binary trees, then\\
$\bullet$ is the root of the strict binary tree in which $\bullet$ is connected via edges to the roots of $t_1$ and $t_2$.\\

Prove that the set of all strict binary trees is countable.
\end{question}

\begin{solution}
{\bf Solution:}\\
{\bf Definitions:}\\
Define a coordinate system $(x,y): \nats\times\nats$ such that $(0,0)$ is the root of the tree.\\
Each level of the tree can possibly have $2^y$ $\bullet$ and will have $2^y$ corresponding $x$ coordinates (starting at 0). Each $x$ coordinate represents a possible place for a $\bullet$ from left to right.\\
The row below the root will have a $y$ coordinate of $1$ and this pattern will continue throughout the depth of the tree.\\\\
$h(a): S\rightarrow \nats$= The height of the binary tree $a$ \\
$b(a, (x, y)): S\times (\nats\times\nats)\rightarrow \{0,1\} =$ If there is a $\bullet$ at the position $(x,y)$ in the tree $a$ function returns $1$ and returns $0$ if not \\
$P_i: \nats\rightarrow\nats = $ The $i$th prime number let $P(0)$ be $2$ and $P(1)$ be $3$ and so forth
The function $F(a)$ is a surjective function from $S\rightarrow \nats$\\
$$F(a): S\rightarrow \nats= \sum_{i=0}^d\left[P_i\sum_{j=0}^{2^i}\left[2^j\cdot b(a,(j,i))\right]\right]$$\\
Using the function $b$ defined above, the inner sum will result in a natural number that when converted to base 2 will have a length of $2^i$. Each digit in this binary string represents a possible location for a bullet (1 if there is one there and 0 if there isn't). Since each place in the binary string has a unique value, there will be a unique corresponding natural number for every possible row of that length.  \\
For every row in the binary tree, the unique numerical value of that row is multiplied by a unique prime number then added together. By the Fundamental theorem of arithmatic, a unique product of primes will result in a unique number. Therefore each number will be unique.\\
Therefore since for each strict binary tree, there is a corresponding natural number. Given an arbitrary natural number that represents a strict binary tree, by factoring the number and converting the occurances of each factor into binary, the underlying binary tree can be determined.
Therefore there is a surjective function from $\nats \rightarrow S$ and by extension $S$ is countable.
\end{solution}
\end{enumerate}
\end{document}
