\documentclass[11pt]{article}
\usepackage{amssymb}
\usepackage{fullpage}
\usepackage{mathrsfs}
\usepackage{comment}
\includecomment{solution}
\includecomment{question}
\setlength{\parindent}{0ex}
\setlength{\parskip}{1ex}
\def\nats {{\mathbb N}}
\def\ints {{\mathbb Z}}
\newcommand{\Implies}{\mbox{ IMPLIES }}
\newcommand{\Or}{\mbox{ OR }}
\newcommand{\And}{\mbox{ AND }}
\newcommand{\Not}{\mbox{NOT }}
\newcommand{\Iff}{\mbox{ IFF }}
\newcommand{\True}{\mbox{T}}
\newcommand{\False}{\mbox{F}}
\newcommand{\Subsets}[1]{\mathscr{P}_{#1}(\{1,\ldots N\})}

% Environment for writing formal proofs======================================
%
% \nl ends line and makes next line a Numbered Line
% \ul ends line and makes next line an Unnumbered Line
% \n increases the level of indentation ("next")
% \p decreases the level of indentation ("previous")
%\firstline  to number the first line of a proof
%
% Example:
%
% 1     let $x \in \nats$ be arbitrary
% 2         let $y = x+1 \in \nats$
% 3         $y > x$; property of $\nats$, 2
% 4      $\exists y \in \nats. (y > x)$; construction 2, 3
% 5  $\forall x \in \nats. \exists y \in \nats. (y > x)$; generalization 1,4
%

% this can be typed as follows:
%
% \begin{formal}
% \firstline          <--- to number the first line of a proof
% \n \label{gen-start} let $x \in \nats$ be arbitrary \nl
% \n let $y = x+1 \in \nats$ \label{defy} \nl
% $y > x$; property of $\nats$, \lref{defy} \label{gt} \nl
% \p $\exists y \in \nats. (y > x)$; \label{inside} construction \lref{defy},\lref{gt} \nl
% \p $\forall x \in \nats. \exists y \in \nats. (y > x)$; generalization \lref{gen-start}, \lref{inside}
% \end{formal}
%
% You can use \labels anywhere in the code and \lref to refer to the line
% number.  (This is done using smartref package.)
%
%to reset the line numbering counter:
%\setcounter{linenum}{0}

\usepackage{smartref} % for referencing the line numbers
\newcounter{linenum}
\addtoreflist{linenum}
\def\codeTabSpace{\hspace*{4mm}}
\newenvironment{formal}%
{\begin{tabbing}%
\codeTabSpace \= \hspace*{20mm} \= \hspace*{20mm} \= \hspace*{20mm} \= \kill%
}%
{\end{tabbing}%
}
\newcounter{ind}
\newcommand{\n}{\addtocounter{ind}{5}\hspace*{5mm}}
\newcommand{\p}{\addtocounter{ind}{-5}\hspace*{-5mm}}
\newcommand{\nl}{\\\stepcounter{linenum}{\scriptsize \arabic{linenum}}\>\hspace*{\value{ind}mm}}
\newcommand{\ul}{\\\>\hspace*{\value{ind}mm}}
\newcommand{\bl}{\\[-1.5mm]\>\hspace*{\value{ind}mm}}
\newcommand{\firstline}{\stepcounter{linenum}{\scriptsize \arabic{linenum}}\>}
\newcommand{\lref}[1]{\linenumref{#1}} % use this to refer to a line number
% End of environment for writing formal proofs=====================================

\begin{document}
\begin{center}
Solutions to\\
{\bf \Large \bf CSC240 Winter 2022 Homework Assignment 3}
\end{center}

\begin{solution}
\noindent
{\bf My name and student number:}

\medskip

\noindent
{\bf The list of people with whom I discussed this homework assignment:}\\
\end{solution}


\begin{question}
A  family $\cal F$ of subsets of $\{1,\ldots n\}$ is a {\em cover} for a subset $S \subseteq \{1,\ldots n\}$ if each element of $S$ is an element of at least one set in $\cal F$.\\
A family $\cal F$ of subsets of $\{1,\ldots n\}$ is {\em 2-cover-free} if, for each set $S \in \cal F$,
no family of two other sets in $\cal F$ is a cover for $S$.\\

For any $e \in \ints^+$ and any subset $X \in {\mathcal P}(\ints^+)$, let MEMBER($e,X$) =  ``$e$ is an element of $X$''.\\
For every  family  $\cal F$ of subsets of $\{1,\ldots n\}$,
the predicate 2CF($\cal F$) = ``$\cal F$ is 2-cover-free'' can be written in predicate logic as
$$\begin{array}{ll}
\forall S \in {\cal F}. \Not \exists X \in {\cal F}. \exists Y \in {\cal F}. & [(\Not (X=S)) \And (\Not(Y=S)) \And \\
& \forall e \in S.({\rm MEMBER}(e,X) \Or {\rm MEMBER}(e,Y)) ].
\end{array}$$

For any positive integer $k\leq N$, let $\Subsets{k}$ denote the set of all
subsets of $\{1,\ldots N\}$ of size at most $k$.\\
A family ${\cal F}$ of subsets of $\{1,\ldots N\}$ is $k$-{\em selective} if, for every set $S \in \Subsets{k}$ and every element
$e \in S$, there is a set $T \in \cal F$ such that $S \cap T = \{e\}$.\\
For example, if $N = 6$, then ${\cal F} = \{\{1,2\},\{3,4\},\{5,6\},\{1,3,5\},\{2,4,6\}\}$ is 2-selective.\\

For every  family  $\cal F$ of subsets of $\{1,\ldots N\}$ and every positive integer $k\leq N$, the predicate\\
SEL(${\cal F},k)$ =  ``$\cal F$ is $k$-selective'' can be written in predicate logic as
$$\begin{array}{ll}
\forall S \in \Subsets{k}. \forall e \in S. \exists T \in {\cal F}. & [{\rm MEMBER}(e,T) \And\\
& \forall d \in S. ({\rm MEMBER}(d,T) \Implies (d=e))]
\end{array}$$

Let $M$ be a Boolean matrix with $n$ rows and $N$ columns.\\
For each $i \in \{1,\ldots,n\}$, let $R_i = \{ j \in \{1,\ldots,N\} \ |\
M[i,j] = 1\}$.\\
Then ${\cal R} = \{ R_i \ |\ i \in \{1,\ldots,n\}\}$ is a family
of sets over $\{1,\ldots,N\}$.\\
Similarly, for each $j \in \{1,\ldots,N\}$, let $C_j = \{ i \in \{1,\ldots,n\} \ |\
M[i,j] = 1\}$.\\
Then ${\cal C} = \{ C_j \ |\ j \in \{1,\ldots,N\}\}$ is a family
of sets over $\{1,\ldots,n\}$.

For example, if
$$M = \left [
\begin{array}{cccc}
0 & 1 & 0 & 1\\
1 & 1 & 0 & 0\\
1 & 0 & 1 & 1
\end{array}
\right ]$$
then ${\cal R} = \{\{2,4\}, \{1,2\}, \{1,3,4\}\}$ and
${\cal C} = \{\{2,3\}, \{1,2\}, \{3\}, \{1,3\}\}$.
\end{question}

\begin{enumerate}
\item
\begin{question}
Give an inforal proof that, if $\cal R$ is 3-selective, then $\cal C$ is 2-cover-free.\\\\
\end{question}

If any set $X\in\Subsets{3}$ containing the element $e$ is a subset of another set in $S\in\Subsets{3}$ where
$S\cap T=e$, then $X\cap T = e$. $T\in{\cal F}$\\
This is because since $X\subset S$, there are no elements in $X$ that are not in $S$ and since all elements in $S$ except $e$ were excluded, the intersection will be the same for both\\\\
$\forall S\in\Subsets{3}.\forall e\in S(\exists T\in {\cal F}.T\cap S = e) \Implies\\
((\forall X\in \Subsets{3}.(X \subset S\And {\rm MEMBER}(e,X)) \Implies (X\cap T) = e$))\\\\

With this, it can be concluded that if all $S\in\Subsets{3}$ of cardinality 3 have corresponding sets in ${\cal F}$ with an intersection of each element within $S$, then the same will be true for all other sets within $\Subsets{3}$. \\
$\forall S\in \Subsets{3}.\forall e\in S.\exists T\in {\cal F}.(|S| = 3 \And S\cap T=e) \Implies \\
(\forall X\in \Subsets{3}.\forall x\in X.\exists Y\in {\cal F}.(|X|< 3 \Implies X\cap Y = e))$\\\\

\null\quad Assume ${\rm SEL}({\cal R}, 3)$.\\
\null\quad For every value $e\in S$ where $S\in \Subsets{3}$ there exists $T\in {\cal R}$ where \\
\null\quad ${\rm MEMBER}(e,T)\And \exists T \in {\cal F}. [{\rm MEMBER}(e,T) \And$\\
\null\quad $\forall d \in S. ({\rm MEMBER}(d,T) \Implies (d=e))]$\\\\


\item
\begin{question}
Formally prove  ${\rm SEL}({\cal R},3) \Implies {\rm 2CF}({\cal C})$.
Each line of your proof should be numbered.
Do only one step each line.
Explicitly state what proof technique is being
applied and say which earlier lines are being used.
Use appropriate indentation. Your proof must be at most 100 lines long.\\\\
\end{question}

\begin{solution}

\end{solution}
\end{enumerate}
\end{document}